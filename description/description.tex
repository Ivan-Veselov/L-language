\documentclass{amsart}
\usepackage{ifxetex}
\ifxetex
  \usepackage{fontspec}
  \usepackage{xunicode}
  \usepackage{xltxtra}
  \usepackage{xecyr}
  \setmainfont[Mapping=tex-text,Ligatures=TeX]{CMU Serif}
  \usepackage{polyglossia}
  \setdefaultlanguage{russian}
\else
  \usepackage[utf8]{inputenc}
  \usepackage[T2A]{fontenc}
  \usepackage[english,russian]{babel}
  \usepackage{concrete}
\fi
\usepackage{amsthm,amsmath,amsfonts,amssymb}
\usepackage{fullpage}
\usepackage{eufrak}
\usepackage{listings}
\usepackage{color}
\usepackage{xcolor}
\usepackage{tikz}
\usepackage{float}
\usetikzlibrary{automata,arrows}
\usepackage{caption}
\usepackage{subcaption}
\usepackage{hyperref}

\newtheorem{problem}{Задача}

\begin{document}

  \definecolor{dkgreen}{rgb}{0,0.6,0}
  \definecolor{gray}{rgb}{0.5,0.5,0.5}
  \definecolor{mauve}{rgb}{0.58,0,0.82}  

  \newcommand{\problemset}[1]{
    
    \begin{center}
      \Large #1
    \end{center}
  }

  \newcommand{\paren}[1]{\left(#1\right)}

  \lstset{ %
    % language=C++,                % the language of the code
    basicstyle=\footnotesize,           % the size of the fonts that are used for the code
    numbers=left,                   % where to put the line-numbers
    numberstyle=\tiny\color{gray},  % the style that is used for the line-numbers
    stepnumber=1,                   % the step between two line-numbers. If it's 1, each line 
                                    % will be numbered
    numbersep=5pt,                  % how far the line-numbers are from the code
    backgroundcolor=\color{white},      % choose the background color. You must add \usepackage{color}
    showspaces=false,               % show spaces adding particular underscores
    showstringspaces=false,         % underline spaces within strings
    showtabs=true,                 % show tabs within strings adding particular underscores
    frame=single,                   % adds a frame around the code
    rulecolor=\color{black!10},        % if not set, the frame-color may be changed on line-breaks within not-black text (e.g. comments (green here))
    tabsize=2,                      % sets default tabsize to 2 spaces
    captionpos=b,                   % sets the caption-position to bottom
    breaklines=true,                % sets automatic line breaking
    breakatwhitespace=false,        % sets if automatic breaks should only happen at whitespace
    title=\lstname,                   % show the filename of files included with \lstinputlisting;
                                    % also try caption instead of title
    keywordstyle=\color{blue},          % keyword style
    commentstyle=\color{dkgreen},       % comment style
    stringstyle=\color{mauve},        % string literal style
    escapeinside={\%*}{*)},            % if you want to add LaTeX within your code
    morekeywords={done, to},              % if you want to add more keywords to the set
  %  deletekeywords={...}              % if you want to delete keywords from the given language
  }

  \begin{tabbing}
\hspace{11cm} \= Студент: \= Иван Веселов \\
  \> Группа: \> SE + ML \\
  \> Дата: \> \today
\end{tabbing}
\hrule
\vspace{1cm}


  Исходный код программы является последовательностью объявлений функций, за
которой следует основной блок кода -- сама программа.
Блок кода и тело каждой функции состоит из инструкций (statement).

Синтаксис для объявления функции выглядит следующим образом:
\begin{lstlisting}
fun functionName(id1, id2, ..., idn) {
    block
}
\end{lstlisting}

Инструкция -- это одно из следующего:
\begin{itemize}
    \item Определение переменной
\begin{lstlisting}
var identifier [:= expression];
\end{lstlisting}
    Допускается опциональное инициализирующее выражение (о которых дальше).\\
    
    \item Цикл while
\begin{lstlisting}
while (expression) {
    block
}
\end{lstlisting}
    
    \item Команда ветвления
\begin{lstlisting}
if (expression) {
    block
}
\end{lstlisting}
    Или
\begin{lstlisting}
if (expression) {
    block
} else {
    block
}
\end{lstlisting}
    
    \item Присваивание
\begin{lstlisting}
identifier := expression;
\end{lstlisting}
    
    \item Команда записи в выходной поток
\begin{lstlisting}
write(expression);
\end{lstlisting}

    \item Команда чтения из входного потока
\begin{lstlisting}
read(identifier);
\end{lstlisting}

    \item Вызов функции
\begin{lstlisting}
functionName(id1, id2, ..., idn);
\end{lstlisting}
\end{itemize}


Выражением (expression) является (в порядке уменьшения приоритета):
\begin{itemize}
    \item Идентификатор\\
    
    \item Числовой литерал\\
    
    \item Мультипликативный оператор
\begin{lstlisting}
* | / | %
\end{lstlisting}
    
    \item Аддитивный оператор
\begin{lstlisting}
+ | -
\end{lstlisting}
    
    \item Оператор сравнения порядка
\begin{lstlisting}
< | > | <= | >=
\end{lstlisting}
    
    \item Оператор сравнения равенства
\begin{lstlisting}
== | !=
\end{lstlisting}
    
    \item Логическая конъюнкция
\begin{lstlisting}
&&
\end{lstlisting}
    
    \item Логическая дизъюнкция
\begin{lstlisting}
||
\end{lstlisting}
\end{itemize}

Выражение также может содержать группирующие скобки.



\end{document}
