Исходный код программы является последовательностью объявлений функций, за
которой следует основной блок кода -- сама программа.
Блок кода и тело каждой функции состоит из инструкций (statement).

Синтаксис для объявления функции выглядит следующим образом:
\begin{lstlisting}
fun functionName(id1, id2, ..., idn) {
    block
}
\end{lstlisting}

Инструкция -- это одно из следующего:
\begin{itemize}
    \item Определение переменной
\begin{lstlisting}
var identifier [:= expression];
\end{lstlisting}
    Допускается опциональное инициализирующее выражение (о которых дальше).\\
    
    \item Цикл while
\begin{lstlisting}
while (expression) {
    block
}
\end{lstlisting}
    
    \item Команда ветвления
\begin{lstlisting}
if (expression) {
    block
}
\end{lstlisting}
    Или
\begin{lstlisting}
if (expression) {
    block
} else {
    block
}
\end{lstlisting}
    
    \item Присваивание
\begin{lstlisting}
identifier := expression;
\end{lstlisting}
    
    \item Команда записи в выходной поток
\begin{lstlisting}
write(expression);
\end{lstlisting}

    \item Команда чтения из входного потока
\begin{lstlisting}
read(identifier);
\end{lstlisting}

    \item Вызов функции
\begin{lstlisting}
functionName(exp1, exp2, ..., expn);
\end{lstlisting}
\end{itemize}


Выражением (expression) является (в порядке уменьшения приоритета):
\begin{itemize}
    \item Вызов функции\\
    
    \item Идентификатор\\
    
    \item Числовой литерал\\
    
    \item Мультипликативный оператор
\begin{lstlisting}
* | / | %
\end{lstlisting}
    
    \item Аддитивный оператор
\begin{lstlisting}
+ | -
\end{lstlisting}
    
    \item Оператор сравнения порядка
\begin{lstlisting}
< | > | <= | >=
\end{lstlisting}
    
    \item Оператор сравнения равенства
\begin{lstlisting}
== | !=
\end{lstlisting}
    
    \item Логическая конъюнкция
\begin{lstlisting}
&&
\end{lstlisting}
    
    \item Логическая дизъюнкция
\begin{lstlisting}
||
\end{lstlisting}
\end{itemize}

Выражение также может содержать группирующие скобки.

